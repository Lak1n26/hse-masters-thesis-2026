\documentclass[12pt,a4paper]{article}

% Пакеты
\usepackage[utf8]{inputenc}
\usepackage[russian]{babel}
\usepackage{hyperref}
\usepackage{enumitem}

% Настройка гиперссылок
\hypersetup{
    colorlinks=true,
    linkcolor=blue,
    urlcolor=blue,
    citecolor=blue
}

\title{\textbf{Нейросетевые подходы к разработке рекомендаций по повышению продаж в товарных корзинах} \\ 
\large (Neural network to up-sell recommendations on shopping carts)}
\author{}
\date{}

\begin{document}

\maketitle

Тип проекта: \textbf{исследовательский}

\section{План работы над ВКР}

\begin{itemize}
    \item 1 декабря: формирование плана работы и обзор литературы.
    \item 20 января: разработка baseline-решений (TopPopular, TopPersonal, EASE, iALS, TIFU-KNN).
    \item 10 марта: основная работа по ВКР (двухуровневая архитектура с бустингом, модели SASRec, BERT4Rec, ReCANet, GPT, RAG).
    \item 10 мая: доработка ВКР.
    \item 20 мая -- 10 июня: защита ВКР.
\end{itemize}

\section{Обзор литературы}

\begin{enumerate}
    \item \url{https://arxiv.org/abs/2401.16433} -- рассмотрена модель Neural Pattern Associator (NPA) -- модель для задачи рекомендаций внутри корзины (within-basket recommendation), предсказывающая товары, которые пользователь может добавить к уже имеющимся в корзине. \textbf{Преимущества}: специализированный подход для задачи дополнения корзины, учитывает контекст текущих товаров. Может выявлять сложные нелинейные зависимости между товарами. \textbf{Недостатки}: ограниченная применимость только к задаче within-basket рекомендаций, требует достаточного количества данных о совместных покупках.
    
    \item \url{https://arxiv.org/abs/2308.01308} -- рассмотрена модель BTBR (Bi-directional Transformer Basket Recommendation) -- модель на основе двунаправленных трансформеров для предсказания следующей корзины покупок на основе истории предыдущих корзин пользователя. \textbf{Преимущества}: механизм внимания позволяет находить долгосрочные зависимости, двунаправленность улучшает понимание контекста. Трансформеры показывают хорошие результаты на последовательных данных. \textbf{Недостатки}: высокая вычислительная сложность трансформеров, требовательность к объему данных для обучения. Модель может быть избыточной для простых сценариев покупок.
    
    \item \url{https://arxiv.org/html/2407.21191v2}, \url{https://arxiv.org/abs/2307.00457} -- рассмотрен генеративный подход к рекомендациям, использующий языковые модели и трансформеры для формулирования задачи рекомендаций как задачи генерации последовательностей. \textbf{Преимущества}: единая архитектура для различных задач рекомендаций, возможность использования предобученных языковых моделей, интерпретируемость через генерацию объяснений. \textbf{Недостатки}: очень высокие требования к вычислительным ресурсам, необходимость больших объемов данных, сложность в оптимизации и контроле качества генерации. Может генерировать нерелевантные рекомендации.
    
    \item \url{https://www.ijcai.org/proceedings/2019/0389.pdf} -- рассмотрена модель Beacon (Basket Sequence Correlation Network) -- модель для предсказания следующей корзины, учитывающая корреляции между последовательностями корзин (correlation-sensitive next basket recommendation). \textbf{Преимущества}: явное моделирование корреляций между товарами внутри корзин и между различными корзинами в последовательности. Использует внимание и свертки для захвата паттернов на разных уровнях (товар-товар, корзина-корзина). \textbf{Недостатки}: ограниченная гибкость архитектуры, может уступать более современным трансформер-подходам. Модель 2019 года, могут быть более свежие улучшения.
    
    \item \url{https://arxiv.org/pdf/1905.03375} -- модель EASE -- простая и эффективная модель для коллаборативной фильтрации на основе линейных автоэнкодеров, оптимизированная для разреженных данных. \textbf{Преимущества}: исключительная простота реализации и скорость работы, замкнутое решение без итеративной оптимизации, отличное качество для бейзлайна. Легко интерпретируется и масштабируется. \textbf{Недостатки}: ограниченная выразительность из-за линейности, не учитывает временную динамику и последовательности, не использует дополнительные признаки товаров или пользователей. Чисто коллаборативный подход.

\end{enumerate}

\section{Особенности моего решения}

\begin{enumerate}
    \item \textbf{Использование свежего кросс-доменного датасета}
    
    В работе используется датасет T-ECD (T-Bank E-Commerce Dataset) -- крупномасштабный датасет кросс-доменных данных российского ритейлера (Т-банк), включающий данные из marketplace, retail и offers доменов. Датасет содержит информацию о транзакциях, корзинах, товарах, брендах и отзывах пользователей.
    
    \begin{itemize}
        \item Публикация на Habr: \href{https://habr.com/ru/companies/tbank/articles/950696/}{habr.com}
        \item Датасет на Hugging Face: \href{https://huggingface.co/datasets/t-tech/T-ECD}{huggingface.co/datasets/t-tech/T-ECD}
    \end{itemize}
    
    \item \textbf{Комплексный бенчмаркинг моделей}
    
    В рамках исследования реализованы и сравнены различные подходы к задаче up-sell рекомендаций:
    
    \begin{itemize}
        \item \textit{Baseline-модели}: TopPopular, TopPersonal, EASE, iALS, TIFU-KNN
        \item \textit{SOTA-модели}: SASRec, BERT4Rec, ReCANet, GPT-based подходы
        \item \textit{Basket-специфичные модели}: NPA, BTBR, Beacon, GenRec
    \end{itemize}
    
    Для каждой модели рассчитаны стандартные метрики качества рекомендаций: NDCG@k, Precision@k, Recall@k, что позволяет провести объективное сравнение подходов.
    
    \item \textbf{Двухуровневая архитектура с бустингом}
    
    Предложена новая двухуровневая архитектура для задачи up-sell рекомендаций:
    
    \begin{itemize}
        \item \textit{Первый уровень (Candidate Generation)}: быстрая генерация множества потенциальных кандидатов из всего каталога товаров с использованием эффективных моделей (EASE, ALS)
        \item \textit{Второй уровень (Ranking)}: точное ранжирование отобранных кандидатов с использованием сложных нейросетевых моделей (трансформеры)
        \item Интеграция бустинговых моделей (LightGBM, CatBoost) для финального ранжирования с учетом множества признаков
    \end{itemize}
    
    Такой подход позволяет сбалансировать качество рекомендаций и скорость инференса, что критично для production-систем.
    
    \item \textbf{RAG-архитектура для up-sell рекомендаций}
    
    Разработана и реализована инновационная RAG (Retrieval-Augmented Generation) архитектура, адаптированная для задачи up-sell рекомендаций на корзинах:
    
    \begin{itemize}
        \item \textit{Retrieval-компонент}: извлечение релевантных товаров и контекстной информации (описания, категории, характеристики) на основе текущего содержимого корзины
        \item \textit{Generation-компонент}: использование языковых моделей для генерации персонализированных рекомендаций с объяснениями
        \item Интеграция мультимодальной информации: текстовые описания товаров, категории, бренды, цены
    \end{itemize}
    
    Данный подход позволяет не только генерировать точные рекомендации, но и предоставлять интерпретируемые объяснения для пользователей.
    
    \item \textbf{Учет кросс-доменной специфики}
    
    Особое внимание уделяется анализу различий в поведении пользователей и эффективности моделей в различных доменах (marketplace vs retail vs offers), что позволяет разработать более гибкие и адаптивные решения.
    
    \item \textbf{Практическая применимость}
    
    Все разработанные решения ориентированы на возможность внедрения в production-системы.
\end{enumerate}

\end{document}



